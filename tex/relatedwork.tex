\section{Related Work}
\label{sec:related}

\todo{Something more like the taxonomy from Phi? Say there are two styles of model}
\todo{Do a grid..}
\todo{Performance Model vs Roofline}
\todo{Power/Runtime models mcpat}
\todo{This Paper}

% NOTE - also mention 3 c's maybe.
The idea that a conceptually simple visual heuristic can deliver worthwhile insights is exemplified by the popular Roofline model \cite{williams:2009aa}. The Roofline model has itself been extended to consider power consumption \cite{choi:2013aa} 


\todo{Work to investigate system max power ..sypo and automate this process.. whatever the other one is}
\todo{System-level Max Power paper}

Models which combine power and timing information exist for codes \todo{Cite McPat, Orion}

System-level Max Power ati



\todo{Analyzing the Energy-Time Trade-Off in High-Performance Computing Applications}

\begin{itemize}
  \item There are models for power/timing
  \item There are analogy models for time
  \item POSE serves to extend this to analogy for power/timing
\end{itemize}

%
%%Relevant bits from Roofline
%  A model need not be perfect, just insightful.
%  For example, the 3Cs model for caches is an analogy.
%  It is not a perfect model, since it ignores potentially important factors like block size, block allocation policy, and block replacement policy.
%  Moreover, it has quirks. For example, a miss can be labeled capacity in one design and conflict in another cache of the same size.
%  Yet, the 3Cs model has been popular for nearly 20 years because it offers insights into the behavior of programs, helping programmers, compiler writers, and architects improve their respective designs.}
%
%    Stochastic analytical models [14][28] and statistical performance models [7][27] can predict program performance on multiprocessors accurately.
%    However, they rarely provide insights into how to improve performance of programs, compilers, or computers [1] or they can be hard to use by non-experts [27].
%    An alternative, simpler approach is bound and bottleneck analysis. Instead of trying to predict performance, it provides [20]

% Relevant bits from energy roofline



