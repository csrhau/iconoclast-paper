\section{Approaches to Modelling}
\label{sec:related}
Performance modelling techniques allow the rapid exploration of ever expanding hardware and software design spaces.
They enable engineers to reason about the performance of their code without the need to aquire measurements for all possible optimisations and configurations.
In this section we present a breakdown of modelling approaches and show where our technique fits in.

\begin{table}
  \centering
  \caption{Modelling Granularity}
  \setlength{\tabcolsep}{10pt}
  \begin{tabular}{lll}
  \toprule
    & \multicolumn{2}{l}{Domain}\\ \cmidrule(){2-3}
  Model Type  & Runtime & Energy \\
    \midrule
  Simulation & SST~\cite{rodrigues:2011aa} & McPAT~\cite{li:2009aa}  \\
  Analytical & Bunt~\cite{bunt:2013aa} & Karkhanis~\cite{karkhanis:2007aa} \\
  Heuristic & Roofline~\cite{williams:2009aa} & \textbf{This Work} \\
  \bottomrule
  \end{tabular}
  \label{tab:approaches}
\end{table}

We divide the performance modelling ecosystem into categories based on their granularity; namely simulations, analytical models and heuristic models.
These approaches have been applied to energy as well as runtime modelling.
\autoref{tab:approaches} lists these categories and provides examples for each. 

\subsubsection{Simulators:} 
Tools like SST~\cite{rodrigues:2011aa} and McPAT~\cite{li:2009aa} operate by executing a simplified representation of the original code whilst gathering data.
This allows them to produce insights which would not be available on a real system.
Although these appraoches are extremely insightful it is often challenging to construct and validate representative simulations.
They also tend to be expensive to run, sometimes even more so than the original code.


\subsubsection{Analytical Models:}  These techniques seek to distil the structure of a program into a parameterised equation.
Analytical performance modelling seeks to 
These models tend to be much faster than 
parameterized 
\subsubsection{Heuristic Models:}
This is the most abstract approach to performance modelling.
Heuristic models do not attempt to faithfully represent every component of a system.
Heuristic models try to


Developer time is a finite resource.

This is the most abstract of the three p
These approaches differ from the o not seek to directly model a system.
Rather, they try to provide insight into some globe
These models do not xxx. Rather, they  capture the essence of a system 
The idea that a conceptually simple visual heuristic can deliver worthwhile insights is exemplified by the popular Roofline model \cite{williams:2009aa}. The Roofline model has itself been extended to consider power consumption \cite{choi:2013aa} 


% NOTE - also mention 3 c's maybe.
aodels which combine power and timing information exist for codes \todo{Cite McPat, Orion}

System-level Max Power ati

POSE is Asymptotic analysis.


\todo{Analyzing the Energy-Time Trade-Off in High-Performance Computing Applications}

%
%%Relevant bits from Roofline
%  A model need not be perfect, just insightful.
%  For example, the 3Cs model for caches is an analogy.
%  It is not a perfect model, since it ignores potentially important factors like block size, block allocation policy, and block replacement policy.
%  Moreover, it has quirks. For example, a miss can be labeled capacity in one design and conflict in another cache of the same size.
%  Yet, the 3Cs model has been popular for nearly 20 years because it offers insights into the behavior of programs, helping programmers, compiler writers, and architects improve their respective designs.}
%
%    Stochastic analytical models [14][28] and statistical performance models [7][27] can predict program performance on multiprocessors accurately.
%    However, they rarely provide insights into how to improve performance of programs, compilers, or computers [1] or they can be hard to use by non-experts [27].
%    An alternative, simpler approach is bound and bottleneck analysis. Instead of trying to predict performance, it provides [20]

% Relevant bits from energy roofline

Each of these categories have their uses and are often used in a complementary fashion.
Heuristic models are easy to construct and guide a developer as to where to look for optimisations.
Analytical models and simulators can then help speed the search for these optimisations.

In summary, POSE represents a preliminary `first cut' modelling technique that identifies inadequate approaches in the early stages of optimisation.i
In so doing it allows developers to focus their efforts where they will be most beneficial. 
