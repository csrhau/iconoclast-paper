\begin{abstract}
Performance engineers are beginning to explore software-level optimisation as a means to reduce the energy consumed when running their codes.
This paper presents POSE, a mathematical and visual modelling tool which highlights the relationship between runtime and power consumption.
POSE allows developers to assess whether power optimisation is worth pursuing for their codes.

We demonstrate POSE by studying the power optimisation characteristics of applications from the Mantevo and Rodinia benchmark suites.
We show that LavaMD has the most scope for CPU power optimisation, with improvements in Energy Delay Squared Product ($\mathbf{ED^2P}$) of up to 30.59\%.
Conversely, MiniMD offers the least scope, with improvements to the same metric limited to 7.60\%.
We also show that no power optimised version of MiniMD operating below 2.3 GHz can match the $\mathbf{ED^2P}$ performance of the original code running at 3.2 GHz.
For LavaMD this limit is marginally less restrictive at 2.2 GHz.
\end{abstract}
