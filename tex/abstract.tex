\begin{abstract}
Performance engineers are beginning to explore software-level optimisation as a means to reduce the energy consumed when running their codes.
This paper presents POSE, a visual heuristic which models the relationship between runtime and power consumption for a particular code and architecture.
POSE allows developers to assess how much improvement can be expected from power optimisations and hence whether they are worth pursuing.

We demonstrate POSE by studying the power optimisation characteristics of codes from the Mantevo and Rodinia benchmark suites.
Our findings indicate that of these codes LavaMD offers the most scope for CPU power optimisation and MiniMD the least.
We then investigate the effect frequency scaling has on the scope for optimising these two codes. 
\end{abstract}
