\begin{abstract}
Performance engineers are beginning to explore software-level power optimisation as a means to reduce the energy consumed by their codes.
This paper presents POSE, an intuitive visual model which captures the relationship between runtime and power consumption for a paticular code and architecture.
Our technique allows developers to assess how much improvement can be expected from power optimisation and hence whether it is worthwhile for their code.

We demonstrate POSE with a CPU power consumption study of two molecular dynamics miniapps.
For MiniMD we find that the potential energy savings from power optimisation are limited to XXX.
We also find that a runtime optimisation of XXXs or more will dominate any possible power optimisation for the $ED^2P$ metric.
For LavaMD these figures are XXX and XXXs respectively.
We conclude that MiniMD offers little scope for power optimisation, whilst LavaMD shows more promise.
\end{abstract}
