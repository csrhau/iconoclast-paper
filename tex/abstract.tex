\begin{abstract}
Performance engineers are beginning to explore software-level optimisation as a means to reduce the energy consumed when running their codes.
This paper presents POSE, a visual heuristic which captures the relationship between runtime and power consumption for a particular code and architecture.
POSE allows developers to assess how much improvement they can expect from power optimisation and hence whether it is worth pursuing.

We validate POSE with a study into the power optimisation characteristics of codes taken from the Mantevo and Rodinia suites.
Our findings indicate that out of these codes MiniMD offers the most scope for CPU power optimisation whilst LavaMD shows the least promise.
In both cases we provide quantitative limits describing the scope each code has for power optimisation as well the benefit this may offer.
The effect frequency scaling has on optimisiation opportunities is also investigated for these two codes. 
\end{abstract}
