\begin{abstract}
Performance engineers are beginning to explore software-level optimisation as a means to reduce the energy consumed when running their codes.
This paper presents POSE, a visual heuristic which captures the relationship between runtime and power consumption for a particular code and architecture.
POSE allows developers to assess how much improvement they can expect from power optimisation and hence whether it is worth pursuing.

We demonstrate POSE by studying the power optimisation characteristics of codes taken from the Mantevo and Rodinia suites.
Our findings indicate that of these codes MiniMD offers the most scope for CPU power optimisation and LavaMD the least.
The effect of frequency scaling on optimisation scope is also investigated for these two codes. 
\end{abstract}
