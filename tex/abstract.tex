\begin{abstract}
Performance engineers are beginning to explore software-level optimisation as a means to reduce the energy consumed when running their codes.
This paper presents POSE, a visual heuristic which captures the relationship between runtime and power consumption for a particular code and architecture.
POSE allows developers to assess how much improvement they can expect from power optimisation and hence whether it is worth pursuing.

We validate POSE with a study into the power optimisation characteristics of codes taken from the Mantevo and Rodinia suites.
We build POSE models for each application and select the codes which are most and least amenable to power optimisation for further investigation.
Our findings indicate that MiniMD offers little scope for CPU power optimisation, whilst LavaMD shows the most promise.
For these codes we provide quantitative limits describing the scope each code has for power optimisation as well the benefit this may offer.
\end{abstract}
