\section{Power Optimization}
\label{sec:optimization}

Energy is the integral of power consumed over time, or simply $E = \bar{P}t$.
As such, reducing the energy cost of a code can be achieved either by shortening its runtime ($t$) through conventional optimizations or reducing the average power consumption ($\bar{P}$) through power optimization.
The POSE heuristic enables performance engineers to compare the potential benefits of each approach and hence focus their efforts to whichever offers the greatest rewards.

Energy cost is a reasonable metric in cases where energy consumption is of paramount importance.
That said, reliance on this metric leaves open the possibility of significant loss of runtime performance, limiting its usefulness in time sensitive domains.
Metrics combining both runtime and energy costs have been developed to address this issue. 
The simplest of these is the Energy-Delay Product (EDP) metric \cite{gonzales:1995aa}, which is defined as follows:
\begin{align}
  EDP &= Energy \times Runtime \nonumber \\
      &= E \times t \nonumber \\
      &= \bar{P} \times t^2
  \label{eq:edp}
\end{align}

Several extensions to EDP have been proposed which assign greater weight to the runtime component to better reflect the demands of high performance computing.
Common examples include energy-delay-suared product ($Et^{2}$) and energy-delay-cubed product ($Et^{3}$).
We refer to this as the $E^mt^n$ family of metrics, which also includes simple power ($E^1t^{-1}$), energy ($E^1t^0$) and time ($E^0t^1$) as members.

Most of the examples within this paper use $Et^2$ as it been shown to be the most suitable of the $E^mt^n$ metrics when considering a fixed micro-architecture \cite{brooks:2000aa}.
That said, POSE is a general purpose heuristic and applies to all members of the $E^mt^n$ group with $m > 0$ and $n \geq 0$, and indeed any metric which increases in line with runtime and energy consumption.
