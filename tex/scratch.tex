\todo{SCRATCH - REMOVE ME}

\todo{Further complication, Clock gating/DVFS leads to false baseline}

\todo{for tighter bounds a proxy app could be developed}

On Frequency...
It is reasonable to expect that different optimisations may have different effects when CPU frequency scaling is taken into effect. Slowing a processor's clock speed down may alter the balance of pressures on various CPU subsystems. With this in mind, it may be the case that some optimisations are effective at some frequencies but not at others. This motivates us to ask the question at what clock frequency does it become categorically impossible to beat the current unoptimised code.

The figures given by POSE also useful as inputs to sensitivity analysis.


\todo{Work to investigate system max power ..sypo and automate this process.. whatever the other one is}
\todo{System-level Max Power paper}

\todo{Analyzing the Energy-Time Trade-Off in High-Performance Computing Applications}

% LUC: Limiting the Unintended Consequences of power scaling on parallel... Steve emailed it to me 27th May
\todo{LUC can be used to justify frequency scaling changes avaiable optimisations}


\todo{Further complication, Halting, Clock gating/DVFS leads to false baseline.}
It would be invalid to use a large proportions of the silicon are temporarily non-functional as a basis for. 
Clock gating electrically isolates processor subsystems which are not in active use.


