\section{Conclusion}
\label{sec:conclusion}
This paper presents POSE, a visual heuristic which helps to identify candidates for power optimisation.
We demonstrate POSE to be a robust analytical model which provides performance engineers with quantitative and actionable insights.

\subsection{Future Work}
Our future work will further validate POSE by applying it to a broader selection of scientific codes running on a range of machines.
The quantitative nature of our technique makes it particularly well suited to comparison studies.
As such, we are particularly keen to investigate the power optimisation opportunities presented by different architectures.

Our ultimate aim is to demonstrate how POSE may be used to identify specific optimisations.
This will involve developing feasible performance envelopes for individual subsystems including memory, filesystems and processors. 
We also intend to profile specific classes of code and establish $P_{min}$ baselines for each.
Doing so would allow POSE to highlight optimisation opportunities at a per-kernel, per-subsystem level and hence facilitate targeted optimisation.
