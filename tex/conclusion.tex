\section{Conclusion}
\label{sec:conclusion}
\noindent
This paper presents POSE, a mathematical and visual modelling tool which captures the trade off between software power consumption and runtime.
POSE provides insights regarding the scope a code has for power optimisation as well as the level of improvement which can be expected.
These insights help developers to determine whether power or runtime optimisation is the best approach for improving the efficiency of a code.

POSE works by partitioning the energy/runtime plane into areas corresponding to runtime and power optimised versions of an initial code with respect to an optimisation metric.
We provide derivations of POSE's boundaries for the Energy Delay Product family of metrics.
We also discuss the various insights our model provides.

We demonstrate POSE by modelling the CPU power consumption of a number of codes taken from the Rodinia and Mantevo benchmark suites.  
Our results illustrate that runtime optimisation is the preferred approach to reducing the energy consumption of MiniMD; power optimisation is limited to improving the $E^1t^2$ of this code by at most 7.60\%.
LavaMD shows more scope for power optimisation, offering improvements of up to 30.59\% in the same metric. 

Our investigation into frequency scaling highlights the ability of POSE to rule out dominated configurations and hence reduce the optimisation search space.
We show that no power optimised version of MiniMD operating at P-states below 2.3 GHz can match the $ED^2P$ performance of the original unoptimised code running at 3.2 GHz.
Once again LavaMD shows more scope for optimisation, with this limit falling at the marginally less restrictive level of 2.2 GHz.

We believe our results are of interest to performance engineers and serve to demonstrate the practical utility of POSE.
POSE is being engineered for inclusion into a well-known state-of-the-art application analytics tool for HPC clusters and applications \textit{[paragraph on this redacted due to anonymity]}.

\subsection*{Future Work}
\noindent
Work is ongoing to develop the hardware and software required to measure power consumption at scale.
This will allow us further validate POSE by applying it to a broader selection of scientific codes running on a range of architectures.
The quantitative nature of our technique makes it particularly well suited to comparison studies.
As such we intend to investigate the power optimisation opportunities presented by a range of different platforms.

Our ultimate aim is to demonstrate how POSE may be used to identify specific optimisations.
This will involve developing feasible performance envelopes for individual subsystems including memory, file systems and processors. 
We also intend to profile specific classes of code and establish $P_{min}$ baselines for each.
Doing so would allow POSE to highlight optimisation opportunities at a per-kernel, per-subsystem level and hence facilitate targeted optimisation.
