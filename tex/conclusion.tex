\section{Conclusion}
\label{sec:conclusion}
This paper presents POSE, a visual heuristic to describe the energy consumption characteristics of a code.
POSE is a robust analytical model capable of providing developers with quantitative and actionable insights.
The main use case for POSE is to help performance engineers to determine whether power or runtime optimisation is the best approach to take optimising software for reduced energy consumption.

We demonstrate POSE with an investigation into CPU power consumption.

The results in this paper illustrate that runtime optimisation is the preferred approach to reducing the energy consumption of MiniMD and LavaMD to a lesser extent.
Our investigation into frequency scaling also highlights the ability of POSE to rule out dominated configurations and hence reduce the optimisation search space.
We believe that both results are of interest to performance engineers and demonstrate the practical utility of POSE.

\subsection{Future Work}
Our future work will further validate POSE by applying it to a broader selection of scientific codes running on a range of machines.
The quantitative nature of our technique makes it particularly well suited to comparison studies.
As such, we are particularly keen to investigate the power optimisation opportunities presented by different architectures.

Our ultimate aim is to demonstrate how POSE may be used to identify specific optimisations.
This will involve developing feasible performance envelopes for individual subsystems including memory, filesystems and processors. 
We also intend to profile specific classes of code and establish $P_{min}$ baselines for each.
Doing so would allow POSE to highlight optimisation opportunities at a per-kernel, per-subsystem level and hence facilitate targeted optimisation.
