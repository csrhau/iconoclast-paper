\section{Investigation}
\label{sec:investigation}


We demonstrate the utility of POSE with an investigation into the CPU power consumption characteristics of several popular benchmarks.
Our findings allow us to identify those codes which have most to gain from power optimization, and to quantify any potential benefits.

The first stage to applying POSE in practice is to gain an understanding of the target platform in order to derive appropriate values for the maximum and minimum power consumption.
While it would be valid to simply rely on manufacturer published limits, these are typically estimates and tend to be overly conservative.
POSE works best when the maximum and minimum power bounds are as tight as possible, and so we rely on measurements rather than estimates.

Current state-of-the-art processors are based on Complimentary Metal Oxide Semiconductor (CMOS) technology and multi- and many-core super-scalar architectures.
The limits of these technologies are fast approaching and research into the next generation of power-efficient architectures and semiconductor fabrication techniques is under way \cite{esmaeilzadeh:2011aa}.

The power draw of CMOS chips can be separated into components as described by \autoref{eq:totpwr}, of which dynamic and leakage power are the most significant.
Dynamic power refers to power consumed as logic gates change state while a processor performs work. 
Leakage power stems from the fact that at very small scales the insulating properties of silicon break down, allowing some current to leak out even when gates remain inactive.
Other forms of power dissipation exist, however their effects are relatively minor \cite{kaxiras:2008aa}.



\todo{for tighter bounds, proxy application could be developed}p

We employ the code given in \autoref{fig:microbench} to derive our baseline.  It consists of a single instruction, performs no computation and places no demand on the memory subsystems. Any non-trivial computation will have a higher activity factor than this minimal micro-benchmark. If the application to be optimized blocks on IO this can be incorporated by measuring the power consumed when the CPU is inactive and adjusting $P_\alpha$ proportionately. We defer measurement of $P_{\beta}$ for now as its precise value is not relevant to the current discussion. 

\todo{as there are two sources of power, we'll look at them seperately}
We first formalize the notion of maximum and minimum power by defining the range of values that activity factor can take whilst running a code in a given P-state as $[\alpha  .. \beta]$ where $0 < \alpha < \beta < 1$, and their associated power draws as $P_{\alpha}$ and $P_{\beta}$ respectively, with $P_{\alpha} < P_{\beta}$. \todo{Point out in other (multi-system) domains the approach is still valid, we just need a sensible max and min.}

\subsection{Experimental Methodology}

CPU energy consumption was measured using Intel's Running Average Power Limit (RAPL) technology~\cite{david:2010aa}. 
We created an augmented version of the unix \texttt{time} binary which output energy consumption figures as well as conventional runtime figures.
The techniques described in \cite{hackenberg:2013aa} were used to ensure measurement accuracy. 


\begin{figure}[ht]                                                               
\centering                                                                      
\lstset{basicstyle=\ttfamily\footnotesize\bfseries, frame=tb} %small bold text, lines top and bottom 
\lstinputlisting[]{lst/alpha_benchmark.c}              
\caption{Baseline Power Micro-Benchmark}                            
\label{fig:microbench}                                                           
\end{figure}  







\todo{As there are two factors which control CPU power consumption we will consider each seperately. }

\begin{equation}
\label{eq:totpwr}
P_{tot} = P_{dyn} + P_{leak} + P_{other}
\end{equation}
\begin{equation} 
\label{eq:dynpwr}
P_{dyn} \propto CV^{2}Af
\end{equation}

\autoref{eq:dynpwr} is a common approximation for dynamic power in which C denotes load capacitance, $V$ the supply voltage, $A$ the activity factor and $f$ the clock frequency.
Of these, only activity factor is directly related to processor workload as it captures the percentage of logic elements which change state each clock cycle.
Conversely, capacitance is a fixed value arising from the wire lengths of on-chip structures.


Historically, dynamic power has been the biggest contributor to $P_{tot}$, however leakage power has been on track to overtake it since the breakdown of Dennard Scaling.  Sub-threshold and gate-oxide leakage dominate total leakage current, and they both increase exponentially as transistors shrink. Process improvements like the introduction of high-k dielectric materials~\cite{jan:2009aa} have kept leakage power in check over the last decade, however there is no avoiding the fact that insulating properties will degrade as transistors get smaller.

\todo{De-yoda, state these effect power}
Processor frequency and supply voltage may also be influenced by workload through the actions of Dynamic Voltage and Frequency Scaling (DVFS) activity.
These properties vary in tandem, taking on predetermined values dependent on a hardware-specific number of fixed P-States.
To paraphrase, DVFS can act to reduce the clock speed of underutilized processors. When this happens, voltage requirements also decrease.

An important feature of the equations governing power draw is that only $P_{dyn}$ is directly influenced by software, in particular due to its inclusion of the $A$ term. Software can also indirectly effect both dynamic and leakage current if it triggers changes to clock frequency and therefore supply voltage through DVFS. \todo{Ensure define dvfs} \todo{reword this bit - no equations now}

The two factors a performance engineer may influence which contribute to power consumption are activity factor and P-State.
To construct our $P_min$ and $P_max$ we must be able to effectively control these factors.
In order to do this, we have devised the following benchmark.


\todo{consider each in turn as a potential source of optimizations.}
%TODO make sure this is explicitly Linpack table in text
\begin{table}
\centering
\footnotesize
\input{tab/tex/fpe_params.tex}
\caption{Feasible Performance Envelope Parameters (W) (2 d.p.)}
\end{table} 


\begin{figure}[t]%
\begin{subfigure}[t]{.5\linewidth}%
\centering%
\begin{tikzpicture}
  \providecommand{\plotwidth}{\linewidth}
  \begin{axis}[
    axis on top,
    axis x line=bottom,
    axis y line=left,
    xlabel={Runtime (\emph{s})},
    ylabel={Energy (\emph{J})},    
    xmin=24, xmax=31,
    ymin=650, ymax=1650,
    width=\plotwidth,
    height=6.8cm,
    legend columns=3,
    legend to name=minimd:legend,
    legend style={/tikz/every even column/.append style={column sep=0.2cm}} % space out columns a bit
    ]

    %% Model Parameters %%
    \pgfmathsetmacro{\baselinepower}{26.876067}
    \pgfmathsetmacro{\rooflinepower}{49.60612}
    \pgfmathsetmacro{\codepower}{27.95947000000066} 
    \pgfmathsetmacro{\codetime}{30.293834}
    \pgfmathsetmacro{\codeenergy}{846.999542908}
    \pgfmathsetmacro{\energyexp}{1.0}
    \pgfmathsetmacro{\timeexp}{2.0}

    % Sadly, pgfplots sucks too much to calculate cube roots
    % These values are calculated with a ruby script in tools
    \pgfmathsetmacro{\blnodex}{29.897382525363728}
    \pgfmathsetmacro{\brnodex}{30.69554258273286}
    \pgfmathsetmacro{\trnodex}{25.0237350439618}
    \pgfmathsetmacro{\tlnodex}{24.373056016397907}

    %% Intermezzo Values %%
    \pgfmathsetmacro{\brnodey}{\brnodex * \baselinepower}
    \pgfmathsetmacro{\blnodey}{\blnodex * \baselinepower}
    \pgfmathsetmacro{\tlnodey}{\tlnodex * \rooflinepower}
    \pgfmathsetmacro{\trnodey}{\trnodex * \rooflinepower}
    \pgfmathsetmacro{\codeenergy}{\codepower * \codetime}
    \pgfmathsetmacro{\baselineenergy}{\baselinepower * \codetime}

    % arguments: code power, code time, x - todo, apparently not supposed to do pgfmathparse
    \pgfmathdeclarefunction{metricbound}{3}{%
      \pgfmathparse{((#1 * #2^3) / #3^2)}%
    }
    \pgfmathdeclarefunction{definitionbound}{3}{%
      \pgfmathparse{((#1 / #2^3) * #3^4)}%
    }

   % BETA ROOFLINE BOUND 
    \addplot[color=printred, very thick, domain=\pgfkeysvalueof{/pgfplots/xmin}:\pgfkeysvalueof{/pgfplots/xmax}] {\rooflinepower * x};
    \addlegendentry{$P_{max}$ Energy Bound}

    % ALPHA BASELINE BOUND 
    \addplot[color=printgreen, very thick, domain=\pgfkeysvalueof{/pgfplots/xmin}:\pgfkeysvalueof{/pgfplots/xmax}] {\baselinepower * x};
    \addlegendentry{$P_{min}$ Energy Bound} 

    \addplot[color=printorange, domain=\trnodex:\brnodex] { metricbound(\codepower, \codetime, x)};
    \addlegendentry{Optimisation Bound}

    \addplot[color=printblue, domain=\blnodex:\codetime] { definitionbound(\codepower, \codetime, x)};
    \addlegendentry{Contribution Bound}

    \addplot[color=printorange, densely dashed, domain=\tlnodex:\blnodex] {metricbound(\baselinepower, \blnodex, x)};
    \addlegendentry{Optimisation Limit}

    % Constant Time (Vertical) dotted line
    \draw[densely dotted] ({axis cs:\codetime,\baselineenergy}) -- ({axis cs:\codetime,\codeenergy});

    \node[circle,fill,inner sep=1pt] at (axis cs:\codetime,\codeenergy) {};
    \node[above right] at (axis cs:\codetime,\codeenergy) {$\theta$};
    
    \node [above] at ({axis cs:\tlnodex, \tlnodey}) {A};
    \node [above] at ({axis cs:\trnodex, \trnodey}) {B};
    \node [below] at ({axis cs:\blnodex, \blnodey}) {C};
    \node [below] at ({axis cs:\codetime,\baselineenergy}) {D};
    \node [below] at ({axis cs:\brnodex, \brnodey}) {E};
 \end{axis}
\end{tikzpicture}
%
\caption{MiniMD}%
\end{subfigure}%
\begin{subfigure}[t]{.5\linewidth}%
\begin{tikzpicture}
  \providecommand{\plotwidth}{\linewidth}
  \begin{axis}[
    axis on top,
    axis x line=bottom,
    axis y line=left,
  	xlabel={Runtime \emph{(s)}},
    ylabel={Energy \emph{(J)}},    
    xmin=48, xmax=75,
    ymin=1200, ymax=4000,
    width=\plotwidth,
    legend to name=lavamd:legend
    ]


    %% Model Parameters %%
    \pgfmathsetmacro{\baselinepower}{26.876067}
    \pgfmathsetmacro{\rooflinepower}{49.60612}
    \pgfmathsetmacro{\codepower}{32.25937199992712} 
    \pgfmathsetmacro{\codetime}{65.640072}
    \pgfmathsetmacro{\codeenergy}{2117.50750075}
    \pgfmathsetmacro{\energyexp}{1.0}
    \pgfmathsetmacro{\timeexp}{2.0}

    % Sadly, pgfplots sucks too much to calculate cube roots
    % These values are calculated with a ruby script in tools
    \pgfmathsetmacro{\blnodex}{61.76450770785698}
    \pgfmathsetmacro{\brnodex}{69.75881800183261}
    \pgfmathsetmacro{\trnodex}{56.869044551106185}
    \pgfmathsetmacro{\tlnodex}{50.35189300975517}
   
    %% Intermezzo Values %%
    \pgfmathsetmacro{\brnodey}{\brnodex * \baselinepower}
    \pgfmathsetmacro{\blnodey}{\blnodex * \baselinepower}
    \pgfmathsetmacro{\tlnodey}{\tlnodex * \rooflinepower}
    \pgfmathsetmacro{\trnodey}{\trnodex * \rooflinepower}
    \pgfmathsetmacro{\codeenergy}{\codepower * \codetime}
    \pgfmathsetmacro{\baselineenergy}{\baselinepower * \codetime}

    % arguments: code power, code time, x - todo, apparently not supposed to do pgfmathparse
    \pgfmathdeclarefunction{metricbound}{3}{%
      \pgfmathparse{((#1 * #2^3) / #3^2)}%
    }
    \pgfmathdeclarefunction{definitionbound}{3}{%
      \pgfmathparse{((#1 / #2^3) * #3^4)}%
    }


   % BETA ROOFLINE BOUND 
    \addplot[color=printred, very thick, domain=\pgfkeysvalueof{/pgfplots/xmin}:\pgfkeysvalueof{/pgfplots/xmax}] {\rooflinepower * x};
    \addlegendentry{$P_{max}$ Energy Bound}

    % ALPHA BASELINE BOUND 
    \addplot[color=printgreen, very thick, domain=\pgfkeysvalueof{/pgfplots/xmin}:\pgfkeysvalueof{/pgfplots/xmax}] {\baselinepower * x};
    \addlegendentry{$P_{min}$ Energy Bound} 

    \addplot[color=printorange, domain=\trnodex:\brnodex] { metricbound(\codepower, \codetime, x)};
    \addlegendentry{Optimisation Bound}

    \addplot[color=printblue, domain=\blnodex:\codetime] { definitionbound(\codepower, \codetime, x)};
    \addlegendentry{Contribution Bound}

    \addplot[color=printorange, densely dashed, domain=\tlnodex:\blnodex] {metricbound(\baselinepower, \blnodex, x)};
    \addlegendentry{Optimisation Limit}

    % Constant Time (Vertical) dotted line
    \draw[densely dotted] ({axis cs:\codetime,\baselineenergy}) -- ({axis cs:\codetime,\codeenergy});

    \node[circle,fill,inner sep=1pt] at (axis cs:\codetime,\codeenergy) {};
    \node[above right] at (axis cs:\codetime,\codeenergy) {$\theta$};
    
    \node [above] at ({axis cs:\tlnodex, \tlnodey}) {A};
    \node [above] at ({axis cs:\trnodex, \trnodey}) {B};
    \node [below] at ({axis cs:\blnodex, \blnodey}) {C};
    \node [below] at ({axis cs:\codetime,\baselineenergy}) {D};
    \node [below] at ({axis cs:\brnodex, \brnodey}) {E};
 \end{axis}
\end{tikzpicture}
%
\caption{LavaMD}%
\end{subfigure}%
\begin{center}%
\ref{minimd:legend}%
\end{center}%
\caption{$Et^2$ POSE for Activity Factor Optimization}%
\label{fig:minimd}%
\end{figure}

\begin{figure}[t]%
\begin{subfigure}[t]{.5\linewidth}%
\centering%
\@ifundefined{pstateminimdtable}{%
  \pgfplotstableread[col sep=comma]{plot/minimd-pstates/data/pstate_power_4_cores.csv}\pstateminimdtable
}{}

\begin{tikzpicture}
  \begin{axis}[
    width=0.95\linewidth,
    axis on top,
    axis x line=bottom,
    axis y line=left,
  	xlabel={Runtime \emph{(s)}},
    ylabel={Energy \emph{(J)}},    
    xmin=0, xmax=60,
    ymin=0, ymax=1500,
    legend columns=2,
    legend to name=minimd-pstate:legend,
    legend style={/tikz/every even column/.append style={column sep=0.2cm}} % space out columns a bit
    ]

    %% Model Parameters %%
    \pgfplotstablegetelem{0}{Runtime}\of{\pstateminimdtable}
    \pgfmathsetmacro{\codetime}{\pgfplotsretval} 
    \pgfplotstablegetelem{0}{Energy}\of{\pstateminimdtable}
    \pgfmathsetmacro{\codeenergy}{\pgfplotsretval} 
    \pgfmathsetmacro{\baselinepower}{13.510238}

    %% Intermezzo Values %%
    \pgfmathsetmacro{\codepower}{\codeenergy / \codetime}

    % arguments: code power, code time, x, n 
    \pgfmathdeclarefunction{metricbound}{4}{%
      \pgfmathparse{((#1 * #2^(#4 + 1)) / #3^#4)}%
    }
    \pgfmathdeclarefunction{definitionbound}{4}{%
      \pgfmathparse{((#1 / #2^(#4 + 1)) * #3^(#4 + 2))}%
    }

    % ALPHA BASELINE BOUND 
    \addplot[color=printgreen, very thick, name path=basebound, domain=\pgfkeysvalueof{/pgfplots/xmin}:\pgfkeysvalueof{/pgfplots/xmax}] {\baselinepower * x};
    \addlegendentry{$P_{min}$ Energy Bound} 


    %% 3.2 GHz start point
    \pgfmathsetmacro{\blnodex}{23.77199310523716}
    \pgfmathsetmacro{\brnodex}{38.6049404590049}
    \addplot[name path=edpdef, draw=none, domain=\blnodex:\codetime+1, forget plot] {definitionbound(\codepower, \codetime, x, 2)};
    \addplot[name path=edpopt, draw=none, domain=\codetime-1:\brnodex, forget plot] {metricbound(\codepower, \codetime, x, 2)};

    \path[name path=edpspace,
      intersection segments={
        of=edpdef and edpopt,
        sequence=A0 -- B1,
      }
      ]; 
    \addplot[blue!20] fill between[of=edpspace and basebound]; 
    \addlegendentry{3.2 GHz POSE}

    %% PState progression
    \addplot[mark=x, black] table[x=Runtime,y=Energy, trim cells=true] {\pstateminimdtable}
      node[pos=0.0, pin=left:3.2 GHz]{}
      node[pos=1.0, pin=95:1.6 GHz]{}
      node[pos=0.5745, pin={[pin distance=0.15cm] above:2.2 GHz}]{}
    ;
    \addlegendentry{P-state Progression}

    %%
    \pgfmathsetmacro{\codetime}{31.160996}
    \pgfmathsetmacro{\codepower}{26.555410070974624}
    \pgfmathsetmacro{\blnodex}{24.876044579784466}
    \addplot[name path=edpdef, gray, densely dashed, domain=\blnodex:\codetime, forget plot] {definitionbound(\codepower, \codetime, x, 2)};

    %%
    \pgfmathsetmacro{\codetime}{32.221644}
    \pgfmathsetmacro{\codepower}{25.19066851461707}
    \pgfmathsetmacro{\blnodex}{26.17914532437151}
    \addplot[name path=edpdef, gray, densely dashed, domain=\blnodex:\codetime, forget plot] {definitionbound(\codepower, \codetime, x, 2)};

    %%
    \pgfmathsetmacro{\codetime}{33.189564}
    \pgfmathsetmacro{\codepower}{24.052689996168674}
    \pgfmathsetmacro{\blnodex}{27.384280155637242}
    \addplot[name path=edpdef, gray, densely dashed, domain=\blnodex:\codetime, forget plot] {definitionbound(\codepower, \codetime, x, 2)};

    %%
    \pgfmathsetmacro{\codetime}{35.548669}
    \pgfmathsetmacro{\codepower}{21.708309894809283}
    \pgfmathsetmacro{\blnodex}{30.35072062021389}
    \addplot[name path=edpdef, gray, densely dashed, domain=\blnodex:\codetime, forget plot] {definitionbound(\codepower, \codetime, x, 2)};

    %%
    \pgfmathsetmacro{\codetime}{36.852917}
    \pgfmathsetmacro{\codepower}{20.791885293639034}
    \pgfmathsetmacro{\blnodex}{31.919904165043228}
    \addplot[name path=edpdef, gray, densely dashed, domain=\blnodex:\codetime, forget plot] {definitionbound(\codepower, \codetime, x, 2)};

    %%
    \pgfmathsetmacro{\codetime}{38.225746}
    \pgfmathsetmacro{\codepower}{19.926273930664426}
    \pgfmathsetmacro{\blnodex}{33.58161699951163}
    \addplot[name path=edpdef, gray, densely dashed, domain=\blnodex:\codetime, forget plot] {definitionbound(\codepower, \codetime, x, 2)};

    %%
    \pgfmathsetmacro{\codetime}{39.720356}
    \pgfmathsetmacro{\codepower}{19.032817706870503}
    \pgfmathsetmacro{\blnodex}{35.432334683311204}
    \addplot[name path=edpdef, gray, densely dashed, domain=\blnodex:\codetime, forget plot] {definitionbound(\codepower, \codetime, x, 2)};

    %%
    \pgfmathsetmacro{\codetime}{41.430395}
    \pgfmathsetmacro{\codepower}{18.26008617586195}
    \pgfmathsetmacro{\blnodex}{37.47190734697387}
    \addplot[name path=edpdef, gray, densely dashed, domain=\blnodex:\codetime, forget plot] {definitionbound(\codepower, \codetime, x, 2)};

    %% 2.2GHz
    \pgfmathsetmacro{\codetime}{43.161913}
    \pgfmathsetmacro{\codepower}{17.552650110758528}
    \pgfmathsetmacro{\blnodex}{39.55555231782452}
    \addplot[name path=edpdef, red, densely dashed, domain=\blnodex:\codetime] {definitionbound(\codepower, \codetime, x, 2)};
    \addlegendentry{Optimisation Cutoff}

    %%
    \pgfmathsetmacro{\codetime}{45.85507}
    \pgfmathsetmacro{\codepower}{16.61222678321067}
    \pgfmathsetmacro{\blnodex}{42.802165661611}
    \addplot[name path=edpdef, gray, densely dashed, domain=\blnodex:\codetime, forget plot] {definitionbound(\codepower, \codetime, x, 2)};

    %%
    \pgfmathsetmacro{\codetime}{49.7305}
    \pgfmathsetmacro{\codepower}{15.67846259337831}
    \pgfmathsetmacro{\blnodex}{47.323406701270244}
    \addplot[name path=edpdef, gray, densely dashed, domain=\blnodex:\codetime, forget plot] {definitionbound(\codepower, \codetime, x, 2)};

    %%
    \pgfmathsetmacro{\codetime}{52.358683}
    \pgfmathsetmacro{\codepower}{15.137478343372388}
    \pgfmathsetmacro{\blnodex}{50.41098720057872}
    \addplot[name path=edpdef, gray, densely dashed, domain=\blnodex:\codetime, forget plot] {definitionbound(\codepower, \codetime, x, 2)};

    %%
    \pgfmathsetmacro{\codetime}{55.340763}
    \pgfmathsetmacro{\codepower}{14.650088958838532}
    \pgfmathsetmacro{\blnodex}{53.866578212248804}
    \addplot[name path=edpdef, gray, densely dashed, domain=\blnodex:\codetime, forget plot] {definitionbound(\codepower, \codetime, x, 2)};

    %%
    \pgfmathsetmacro{\codetime}{58.637567}
    \pgfmathsetmacro{\codepower}{14.09304050763225}
    \pgfmathsetmacro{\blnodex}{57.81786383949271}
    \addplot[name path=edpdef, gray, densely dashed, domain=\blnodex:\codetime, forget plot] {definitionbound(\codepower, \codetime, x, 2)};


 \end{axis}
\end{tikzpicture}
%
\caption{MiniMD}%
\end{subfigure}%
\begin{subfigure}[t]{.5\linewidth}%
\@ifundefined{pstatelavamdtable}{%
  \pgfplotstableread[col sep=comma]{plot/lavamd-pstates/data/pstate_power_4_cores.csv}\pstatelavamdtable
}{}

\begin{tikzpicture}
  \begin{axis}[
    width=0.95\linewidth,
    axis on top,
    axis x line=bottom,
    axis y line=left,
  	xlabel={Runtime \emph{(s)}},
    ylabel={Energy \emph{(J)}},    
    xmin=0, xmax=140,
    ymin=0, ymax=3800,
    legend columns=2,
    legend to name=lavamd-pstate:legend,
    legend style={/tikz/every even column/.append style={column sep=0.2cm}} % space out columns a bit
    ]

    %% Model Parameters %%
    \pgfplotstablegetelem{0}{Runtime}\of{\pstatelavamdtable}
    \pgfmathsetmacro{\codetime}{\pgfplotsretval} 
    \pgfplotstablegetelem{0}{Energy}\of{\pstatelavamdtable}
    \pgfmathsetmacro{\codeenergy}{\pgfplotsretval} 
    \pgfmathsetmacro{\baselinepower}{13.510238}

    %% Intermezzo Values %%
    \pgfmathsetmacro{\codepower}{\codeenergy / \codetime}

    % arguments: code power, code time, x, n 
    \pgfmathdeclarefunction{metricbound}{4}{%
      \pgfmathparse{((#1 * #2^(#4 + 1)) / #3^#4)}%
    }
    \pgfmathdeclarefunction{definitionbound}{4}{%
      \pgfmathparse{((#1 / #2^(#4 + 1)) * #3^(#4 + 2))}%
    }

    % ALPHA BASELINE BOUND 
    \addplot[color=green, name path=basebound, domain=\pgfkeysvalueof{/pgfplots/xmin}:\pgfkeysvalueof{/pgfplots/xmax}] {\baselinepower * x};
    \addlegendentry{$P_{min}$ Energy Bound} 


    %% 3.2 GHz start point
    \pgfmathsetmacro{\brnodex}{87.73374843784023}
    \pgfmathsetmacro{\blnodex}{49.11016716922632}
    \addplot[name path=edpdef, draw=none, domain=\blnodex:\codetime+1, forget plot] {definitionbound(\codepower, \codetime, x, 2)};
    \addplot[name path=edpopt, draw=none, domain=\codetime-1:\brnodex, forget plot] {metricbound(\codepower, \codetime, x, 2)};

    \path[name path=edpspace,
      intersection segments={
        of=edpdef and edpopt,
        sequence=A0 -- B1,
      }
      ]; 
    \addplot[blue!20] fill between[of=edpspace and basebound]; 
    \addlegendentry{3.2 GHz POSE}

    %% PState progression
    \addplot[mark=x, black] table[x=Runtime,y=Energy, trim cells=true] {\pstatelavamdtable}
      node[pos=0.0, pin=left:3.2 GHz]{}
      node[pos=1.0, pin=95:1.6 GHz]{}
      node[pos=0.636, pin={[pin distance=0.15cm] above:2.1 GHz}]{}
    ;
    \addlegendentry{P-State Progression}


    %%
    \pgfmathsetmacro{\codetime}{67.788671}
    \pgfmathsetmacro{\codepower}{30.534496154969613}
    \pgfmathsetmacro{\blnodex}{51.65525781584337}
    \addplot[name path=edpdef, gray, densely dashed, domain=\blnodex:\codetime, forget plot] {definitionbound(\codepower, \codetime, x, 2)};

    %%
    \pgfmathsetmacro{\codetime}{69.909725}
    \pgfmathsetmacro{\codepower}{28.972621505806238}
    \pgfmathsetmacro{\blnodex}{54.21207070239038}
    \addplot[name path=edpdef, gray, densely dashed, domain=\blnodex:\codetime, forget plot] {definitionbound(\codepower, \codetime, x, 2)};

    %%
    \pgfmathsetmacro{\codetime}{72.820343}
    \pgfmathsetmacro{\codepower}{27.23156518227331}
    \pgfmathsetmacro{\blnodex}{57.647815146592066}
    \addplot[name path=edpdef, gray, densely dashed, domain=\blnodex:\codetime, forget plot] {definitionbound(\codepower, \codetime, x, 2)};

    %%
    \pgfmathsetmacro{\codetime}{78.83738}
    \pgfmathsetmacro{\codepower}{24.363863791516156}
    \pgfmathsetmacro{\blnodex}{64.76958711476986}
    \addplot[name path=edpdef, gray, densely dashed, domain=\blnodex:\codetime, forget plot] {definitionbound(\codepower, \codetime, x, 2)};

    %%
    \pgfmathsetmacro{\codetime}{80.443575}
    \pgfmathsetmacro{\codepower}{23.569310575766927}
    \pgfmathsetmacro{\blnodex}{66.82363092159444}
    \addplot[name path=edpdef, gray, densely dashed, domain=\blnodex:\codetime, forget plot] {definitionbound(\codepower, \codetime, x, 2)};

    %%
    \pgfmathsetmacro{\codetime}{83.583275}
    \pgfmathsetmacro{\codepower}{22.42571518045925}
    \pgfmathsetmacro{\blnodex}{70.59245458487962}
    \addplot[name path=edpdef, gray, densely dashed, domain=\blnodex:\codetime, forget plot] {definitionbound(\codepower, \codetime, x, 2)};

    %%
    \pgfmathsetmacro{\codetime}{87.25231}
    \pgfmathsetmacro{\codepower}{21.415643975500476}
    \pgfmathsetmacro{\blnodex}{74.83203474018039}
    \addplot[name path=edpdef, gray, densely dashed, domain=\blnodex:\codetime, forget plot] {definitionbound(\codepower, \codetime, x, 2)};

    %%
    \pgfmathsetmacro{\codetime}{90.581583}
    \pgfmathsetmacro{\codepower}{20.54295973167084}
    \pgfmathsetmacro{\blnodex}{78.77224707386881}
    \addplot[name path=edpdef, gray, densely dashed, domain=\blnodex:\codetime, forget plot] {definitionbound(\codepower, \codetime, x, 2)};

    %%
    \pgfmathsetmacro{\codetime}{95.357508}
    \pgfmathsetmacro{\codepower}{19.637888418812288}
    \pgfmathsetmacro{\blnodex}{84.1803958382032}
    \addplot[name path=edpdef, gray, densely dashed, domain=\blnodex:\codetime, forget plot] {definitionbound(\codepower, \codetime, x, 2)};

    %%
    \pgfmathsetmacro{\codetime}{99.502899}
    \pgfmathsetmacro{\codepower}{18.820670933416725}
    \pgfmathsetmacro{\blnodex}{89.0932975465577}
    \addplot[name path=edpdef, red, densely dashed, domain=\blnodex:\codetime] {definitionbound(\codepower, \codetime, x, 2)};
    \addlegendentry{Optimization Boundary}
    %%
    \pgfmathsetmacro{\codetime}{109.970804}
    \pgfmathsetmacro{\codepower}{17.334285425429826}
    \pgfmathsetmacro{\blnodex}{101.20370662532102}
    \addplot[name path=edpdef, gray, densely dashed, domain=\blnodex:\codetime, forget plot] {definitionbound(\codepower, \codetime, x, 2)};

    %%
    \pgfmathsetmacro{\codetime}{115.469869}
    \pgfmathsetmacro{\codepower}{16.703561489274747}
    \pgfmathsetmacro{\blnodex}{107.58539357794623}
    \addplot[name path=edpdef, gray, densely dashed, domain=\blnodex:\codetime, forget plot] {definitionbound(\codepower, \codetime, x, 2)};

    %%
    \pgfmathsetmacro{\codetime}{123.182269}
    \pgfmathsetmacro{\codepower}{16.060352379123653}
    \pgfmathsetmacro{\blnodex}{116.28334319474206}
    \addplot[name path=edpdef, gray, densely dashed, domain=\blnodex:\codetime, forget plot] {definitionbound(\codepower, \codetime, x, 2)};

    %%
    \pgfmathsetmacro{\codetime}{130.924339}
    \pgfmathsetmacro{\codepower}{15.486518591474423}
    \pgfmathsetmacro{\blnodex}{125.09985043035239}
    \addplot[name path=edpdef, gray, densely dashed, domain=\blnodex:\codetime, forget plot] {definitionbound(\codepower, \codetime, x, 2)};
  \end{axis}
\end{tikzpicture}
%
\caption{LavaMD}%
\end{subfigure}%
\begin{center}%
\ref{minimd-pstate:legend}%
\end{center}%
\caption{$Et^2$ POSE for P-State Optimization}%
\label{fig:pstates}%
\end{figure}%

\subsection{Results}

\begin{table}
\centering
\input{tab/tex/minimd_pose}
\caption{MiniMD POSE, 4 cores at 3.2 GHz (2 d.p.)}
\label{tab:minimd_pose}
\end{table} 



The figures in \autoref{tab:minimd_pose} allow us to state that the upper limit of energy to be saved from power optimization alone for MiniMD running on our target platform is is 32.82J.
The maximum improvement in $Et^2$ is 59072.00.
All codes with runtime in excess of 30.695542582732617 have higher $Et^2$.
This means we can trade at most 0.40s in exchange for better power performance.
A runtime optimization of 4.16s, or 1.16x is guaranteed to beat $\theta$ in terms of $Et^2$. 
A runtime optimization of 4.84s, or 1.19x, is guaranteed to beat any power optimized version of $\theta$ in terms of $Et^2$ 


\begin{table}
\centering
\input{tab/tex/lavamd_pose}
\caption{LavaMD POSE, 4 cores at 3.2 GHz (2 d.p.)}
\label{tab:lavamd_pose}
\end{table} 

\autoref{tab:lavamd_pose} shows that for LavaMD The upper limit of energy to be saved from power optimization alone is 353.36J.
The maximum improvement in $Et^2$ is 2790924.00.
All codes with runtime in excess of 69.75881800188515 have higher $Et^2$.
This means we can trade at most 4.12s in exchange for better power performance.
A runtime optimization of 6.26s, or 1.11x is guaranteed to beat $\theta$ in terms of $Et^2$. 
A runtime optimization of 13.06s, or 1.25x, is guaranteed to beat any power optimized version of $\theta$ in terms of $Et^2$ 

\subsection{Discussion}
The figures produced by POSE are all upper bounds, and the benefits of power optimization will be more modest in practice. Even so, these figures are useful as they allow performance engineers to make informed decisions about where best to focus their efforts. If they consider a $1.03 \times$ speed up to be more achievable than up to the maximum $1.17\times$ reduction in activity factor then they can proceed to apply conventional optimizations safe in the knowledge that overall performance will improve despite any increases in activity factor.

If a performance engineer decides the benefits of power optimization are worth pursuing after applying POSE, the question still remains as to how he should go about searching for those optimizations.

