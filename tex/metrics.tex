\section{metrics}
\label{sec:metrics}
Historically, runtime has been the predominant metric used in performance critical domains whilst energy has largely been confined to embedded systems.
That said, metrics which capture the importance of power as well as time are gaining prominence as the power requirements of scientific computing continue to rise.

%\begin{figure}
%\centering
%\usepgfplotslibrary{fillbetween}

%Fix Area legend to not draw surrounding box
%Taken from http://tex.stackexchange.com/questions/99861/remove-border-around-area-legend-rectangle
\pgfplotsset{
    /pgfplots/area legend/.style={%
        /pgfplots/legend image code/.code={%
            \fill[##1] (0cm,-0.1cm) rectangle (0.6cm,0.1cm);
        }%
    },
}

\begin{tikzpicture}
  \begin{axis}[ticks = none, 
    axis on top,
    axis x line=bottom,
    axis y line=left,
  	xlabel={Runtime \emph{(s)}},
    ylabel={Energy \emph{(J)}},    
    xmin=0, xmax=50,
    ymin=0, ymax=3300,
    width=\linewidth,
    legend style={legend pos=north west}
    ]

    %% Model Parameters %%
    \pgfmathsetmacro{\baselinepower}{30} % NOP code
    \pgfmathsetmacro{\rooflinepower}{60}
    \pgfmathsetmacro{\codepower}{(\baselinepower*3 + \rooflinepower*4) / 7}
    \pgfmathsetmacro{\codetime}{30}
    % Sadly, pgfplots sucks too much to calculate cube roots
    \pgfmathsetmacro{\anodex}{26.20741}
    \pgfmathsetmacro{\anodey}{\anodex * \baselinepower}
    \pgfmathsetmacro{\cnodex}{34.34143}
    \pgfmathsetmacro{\cnodey}{\cnodex * \baselinepower}
    \pgfmathsetmacro{\tnodex}{27.25681}
 
    %% Intermezzo Values %%
    \pgfmathsetmacro{\codeenergy}{\codepower * \codetime}
    \pgfmathsetmacro{\baselineenergy}{\baselinepower * \codetime}
    \pgfmathsetmacro{\rooflineenergy}{\rooflinepower * \codetime}
    \pgfmathsetmacro{\lowdisplayline}{(2 * \baselinepower + \codepower) / 3}
    \pgfmathsetmacro{\highdisplayline}{(1 * \rooflinepower + 1 * \codepower) / 2}
    \pgfmathsetmacro{\rooflinetime}{\codeenergy/\rooflinepower}
    \pgfmathsetmacro{\baselinetime}{\codeenergy/\baselinepower}

    % arguments: code power, code time, x - todo, apparently not supposed to do pgfmathparse
    \pgfmathdeclarefunction{metricbound}{3}{%
      \pgfmathparse{((#1 * #2^3) / #3^2)}%
    }
    \pgfmathdeclarefunction{definitionbound}{3}{%
      \pgfmathparse{((#1 / #2^3) * #3^4)}%
    }
     \pgfmathdeclarefunction{optimizationlimits}{3}{%
      \pgfmathparse{(min(metricbound(#1, #2, #3), definitionbound(#1, #2, #3)))}
    }

    % BETA ROOFLINE BOUND
    \addplot[color=red, domain=\pgfkeysvalueof{/pgfplots/xmin}:\pgfkeysvalueof{/pgfplots/xmax}] {\rooflinepower * x};
    \addlegendentry{$P_{max}$ Energy Bound}

    %const power diagonal
    \addplot[color=darkgray, densely dashed, forget plot, %forget plot prevents legend entry
            domain=\pgfkeysvalueof{/pgfplots/xmin}:\pgfkeysvalueof{/pgfplots/xmax}] {\codepower * x}; 

    % ALPHA BASELINE BOUND 
    \addplot[color=green, domain=\pgfkeysvalueof{/pgfplots/xmin}:\pgfkeysvalueof{/pgfplots/xmax}] {\baselinepower * x};
    \addlegendentry{$P_{min}$ Energy Bound} 

    %Runtime Optimization
    \addplot[area legend, fill=blue, fill opacity=0.3, draw=none] coordinates { 
      (\codetime,\rooflineenergy)
      (\codetime,\baselineenergy)
    (0,0)};
    \addlegendentry{Runtime Optimization}

    %Power Optimization
    \addplot[area legend, fill=red, fill opacity=0.2, draw=none] coordinates {
                           (\codetime,\codeenergy)
                           (\pgfkeysvalueof{/pgfplots/xmax},\pgfkeysvalueof{/pgfplots/xmax}*\codepower)
                           (\pgfkeysvalueof{/pgfplots/xmax}, \pgfkeysvalueof{/pgfplots/xmax}*\baselinepower)
                           (0,0)
                         } ;
    \addlegendentry{Power Optimization}
  
    %Energy Optimization
    \addplot[area legend, style={pattern=north west lines, pattern color=gray,
                                 draw=none}] coordinates { 
      (\rooflinetime, \codeenergy)
      (\baselinetime, \codeenergy)
      (0, 0) };
    \addlegendentry{Energy Optimization}



    % Constant Time, Energy Dashes
    %vertical
    \draw[densely dotted] ({axis cs:\codetime,\baselineenergy}) -- ({axis cs:\codetime,\rooflineenergy});
    %horizontal
    \draw[densely dotted] ({axis cs:\rooflinetime,\codeenergy}) -- ({axis cs:\baselinetime,\codeenergy});
   

    \node[circle,fill,inner sep=2pt] at (axis cs:\codetime,\codeenergy) {};
    \node[below right] at (axis cs:\codetime,\codeenergy) {$\theta$};


 \end{axis}
\end{tikzpicture}

%\caption{Feasible Performance Envelope}
%\label{fig:motivation}
%\end{figure}
%
Power consumption is a seemingly obvious choice of metric in energy sensitive domains as it captures the rate at which energy is consumed.
The ability to detect power hungry sections of code would clearly help identify candidates for optimization.
That said, power is rarely an appropriate metric to optimize for.
Optimizations to reduce power draw can in principle lead to increases in runtime and in turn greater total energy consumption.
The upper right shaded sector of \autoref{fig:motivation} corresponds to this case.

Energy to completion is a reasonable metric when energy consumption is of paramount importance.
It is often used in domains like mobile robotics and sensor where available energy is severely restricted.
That said, whilst energy is becoming more important in domains like scientific computing, it is not the only limiting factor. 
Reliance on energy as a metric leaves open the possibility of unacceptable loss of runtime performance.

In practice the most useful metrics are those which combine both energy and runtime.
The simplest of these is the Energy-Delay Product (EDP) metric \cite{gonzales:1995aa}, which assigns an equal weighting to both runtime and energy consumption and is defined as follows:
\begin{align}
  EDP = Energy \times Runtime \nonumber \\
      \Leftrightarrow Power \times Runtime^{2} 
  \label{eq:edp}
\end{align}

Several extensions to EDP have been proposed which assign greater weight to the runtime component to better reflect the demands of high performance computing.
Common examples include energy-delay-squared product ($ED^{2}$) and energy-delay-cubed product ($ED^{3}$).
We refer to this as the $E^mD^n$ family of metrics, which also includes simple power ($E^1D^{-1}$), energy ($E^1D^0$) and time ($E^0D^1$) as members.
It has been argued that that $ED^{2}P$ is most suited when considering a fixed micro-architecture \cite{brooks:2000aa}, however our work applies to all members of this group with $m > 0$ and $n \geq 0$.
