\section{Introduction} \noindent
Advances in processor design have delivered improvements in CPU performance for decades. As physical limits are reached, however, refinements to the same basic technologies are beginning to show diminishing returns \cite{esmaeilzadeh:2011aa}. 
One side-effect of this is an unsustainable rise in system power use, which the US Department of Energy has identified as a primary constraint for exascale systems \cite{shalf:2011aa}.

Hardware manufacturers are increasingly prioritising energy efficiency in processor designs~\cite{kurd:2014aa}. 
Research suggests that software modifications will be required to fully exploit the resulting improvements in modern processors~\cite{shao:2013aa}.
The development of new energy-aware performance engineering tools and techniques will help developers to identify and captialize on this new class of optimisation.

In this paper we present the Power Optimised Software Envelope (POSE).
POSE is a mathematical and visual modelling tool which provides insights into the energy consumption characteristics of a code.
Our work helps performance engineers understand whether power or runtime optimisation is the best strategy for improving the energy efficiency of their codes.

\noindent
The contributions made in this work are:
\begin{itemize}
  \item We introduce POSE, providing derivations for its constituent boundaries and an overview of the insights it provides;
  \item We show how POSE can be targeted to specific platforms and use-cases. 
        Specifically, we investigate the trade-offs between runtime and CPU power consumption;
  \item We use POSE to study codes from the Mantevo and Rodinia benchmark suites.
        We assess the potential benefits of power optimisation for each code, showing that LavaMD offers the most scope for power optimisation while MiniMD offers the least;
  \item Finally, we investigate how opportunities for power optimisation vary in response to frequency scaling for these two codes.
\end{itemize}

\noindent
The remainder of this paper is structured as follows:
\autoref{sec:related} presents a survey of related work; 
\autoref{sec:pose} details the construction of POSE along with the various insights it provides;
\autoref{sec:investigation} demonstrates our new modelling tool with a study into the CPU power optimisation opportunities presented by a range of benchmark applications;
and finally \autoref{sec:conclusion} concludes the paper and describes future research.
