\section{Introduction}
Advances in processor design have delivered improvements in CPU performance for decades. As physical limits are reached, however, refinements to the same basic technologies are beginning to show diminishing returns \cite{esmaeilzadeh:2011aa}. One side-effect of this is an unsustainable rise in system power use, which the US Department of Energy has identified as a primary constraint for exascale systems \cite{shalf:2011aa}.

Hardware manufacturers are increasingly prioritising energy efficiency in processor designs~\cite{kurd:2014aa}. 
Research suggests that software modifications will be required to fully exploit the energy efficiency improvements in modern processors~\cite{shao:2013aa}.
Energy-aware optimisation of code 

In this paper we present the Power Optimised Software Envelope (POSE) technique.
POSE is a visual model which provides insights into the energy consumption characteristics of a code.
Our work helps performance engineers decide whether to pursue power or runtime optimisations when attempting to improve the energy efficiency of their code.

\medskip \noindent
The contributions made in this work are:
\begin{itemize}
  \item We introduce POSE, providing derivations for its constituent boundaries and an overview of the various insights it offers.
  \item We show how POSE, which is a general technique, can be targeted to specific platforms and use-cases. 
  \begin{itemize}
    \item Specifically, we develop a process and supporting toolchain to investigate the tradeoffs between runtime and CPU power consumption.
  \end{itemize}
  \item We validate POSE by undertaking a comparative study of two molecular dynamics miniapps, MiniMD and LavaMD.
  \begin{itemize}
    \item We apply the POSE heuristic to assess the potential benefits offered by power optimisation for each code.
    \item We also investigate how opportunities for power optimisation vary in response to frequency scaling for both codes.
    \item We show that there is more scope for power optimisation with LavaMD than there is for MiniMD.
  \end{itemize}
\end{itemize}

The remainder of this paper is structured as follows: \autoref{sec:related} introduces existing work in the area. 
\autoref{sec:optimisation} then describes the problem domain and provides a summary of key concepts and metrics.
\autoref{sec:pose} details the construction of our heuristic before \autoref{sec:insights} lists the various insights it provides.
We validate our model in \autoref{sec:investigation} by using it to study the CPU power optimisation opportunities presented by two molecular dynamics miniapps. 
We end by presenting our conclusions and future work in \autoref{sec:conclusion}.
