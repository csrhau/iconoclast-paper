\section{Introduction}
Advances in processor design have delivered improvements in CPU performance for decades. As physical limits are reached, however, refinements to the same basic technologies are beginning to show diminishing returns \cite{esmaeilzadeh:2011aa}. One side-effect of this is an unsustainable rise in system power use, which the US Department of Energy has identified as a primary constraint for exascale systems \cite{shalf:2011aa}.

Hardware manufacturers are increasingly prioritising energy efficiency in processor designs~\cite{kurd:2014aa}. 
Research suggests that software modifications will be required to fully exploit the resulting improvements in modern processors~\cite{shao:2013aa}.

In this paper we present the Power Optimised Software Envelope (POSE) technique.
POSE is a visual model which provides insights into the energy consumption characteristics of a code.
Our work helps performance engineers understand whether power or runtime optimisation is the best strategy for improving the energy efficiency of their code.

\medskip \noindent
The contributions made in this work are:
\begin{itemize}
  \item We introduce the POSE model, providing derivations for its constituent boundaries and an overview of the various insights it offers.
  \item We show how POSE, which is a general technique, can be targeted to specific platforms and use-cases. 
  \begin{itemize}
    \item Specifically, we develop a process and supporting toolchain to investigate the tradeoffs between runtime and CPU power consumption.
  \end{itemize}
  \item We validate POSE by undertaking a comparative study of codes from the Mantevo and Rodinia application suites.
  \begin{itemize}
    \item We apply the POSE heuristic to assess the potential benefits of power optimisation for each code.
    \item We show that LavaMD offers the most scope for power optimisation while MiniMD offers the least.
    \item We investigate how opportunities for power optimisation vary in response to frequency scaling for these two codes.
  \end{itemize}
\end{itemize}

The remainder of this paper is structured as follows: \autoref{sec:approaches} lists different approaches to performance modelling and explains where POSE fits in this landscape.
\autoref{sec:optimisation} then provides a summary of key concepts and metrics.
\autoref{sec:pose} details the construction of POSE before \autoref{sec:insights} describes the various insights it provides.
We validate our model in \autoref{sec:investigation} by using it to study the CPU power optimisation opportunities presented by a range of proxy applications. 
We end with conclusions and future work in \autoref{sec:conclusion}.
