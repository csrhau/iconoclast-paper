\usepgfplotslibrary{fillbetween}

%Fix Area legend to not draw surrounding box
%Taken from http://tex.stackexchange.com/questions/99861/remove-border-around-area-legend-rectangle
\pgfplotsset{
    /pgfplots/area legend/.style={%
        /pgfplots/legend image code/.code={%
            \fill[##1] (0cm,-0.1cm) rectangle (0.6cm,0.1cm);
        }%
    },
}

\begin{tikzpicture}
  \begin{axis}[ticks = none, 
    axis on top,
    axis x line=bottom,
    axis y line=left,
  	xlabel={Runtime \emph{(s)}},
    ylabel={Energy \emph{(J)}},    
    xmin=0, xmax=50,
    ymin=0, ymax=3800,
    width=0.9\linewidth,
    legend style={legend pos=north west}
    ]

    %% Model Parameters %%
    \pgfmathsetmacro{\baselinepower}{30} % NOP code
    \pgfmathsetmacro{\rooflinepower}{60}
    \pgfmathsetmacro{\codepower}{(\baselinepower*3 + \rooflinepower*4) / 7}
    \pgfmathsetmacro{\codetime}{30}
    % Sadly, pgfplots sucks too much to calculate cube roots
    \pgfmathsetmacro{\anodex}{26.20741}
    \pgfmathsetmacro{\anodey}{\anodex * \baselinepower}
    \pgfmathsetmacro{\cnodex}{34.34143}
    \pgfmathsetmacro{\cnodey}{\cnodex * \baselinepower}
    \pgfmathsetmacro{\tnodex}{27.25681}
 
    %% Intermezzo Values %%
    \pgfmathsetmacro{\codeenergy}{\codepower * \codetime}
    \pgfmathsetmacro{\baselineenergy}{\baselinepower * \codetime}
    \pgfmathsetmacro{\rooflineenergy}{\rooflinepower * \codetime}
    \pgfmathsetmacro{\lowdisplayline}{(2 * \baselinepower + \codepower) / 3}
    \pgfmathsetmacro{\highdisplayline}{(1 * \rooflinepower + 1 * \codepower) / 2}
    \pgfmathsetmacro{\rooflinetime}{\codeenergy/\rooflinepower}
    \pgfmathsetmacro{\baselinetime}{\codeenergy/\baselinepower}

    % arguments: code power, code time, x - todo, apparently not supposed to do pgfmathparse
    \pgfmathdeclarefunction{metricbound}{3}{%
      \pgfmathparse{((#1 * #2^3) / #3^2)}%
    }
    \pgfmathdeclarefunction{definitionbound}{3}{%
      \pgfmathparse{((#1 / #2^3) * #3^4)}%
    }
     \pgfmathdeclarefunction{optimisationlimits}{3}{%
      \pgfmathparse{(min(metricbound(#1, #2, #3), definitionbound(#1, #2, #3)))}
    }

    % BETA ROOFLINE BOUND
    \addplot[color=red, forget plot, domain=\pgfkeysvalueof{/pgfplots/xmin}:\pgfkeysvalueof{/pgfplots/xmax}] {\rooflinepower * x};

    %const power diagonal
    \addplot[color=darkgray, densely dashed, forget plot, %forget plot prevents legend entry
             domain=\pgfkeysvalueof{/pgfplots/xmin}:\pgfkeysvalueof{/pgfplots/xmax}] {\codepower * x}; 

    % ALPHA BASELINE BOUND 
    \addplot[color=green, forget plot, domain=\pgfkeysvalueof{/pgfplots/xmin}:\pgfkeysvalueof{/pgfplots/xmax}] {\baselinepower * x};

    %Runtime Optimisation
    \addplot[area legend, fill=blue, fill opacity=0.3, draw=none] coordinates { 
      (\codetime,\rooflineenergy)
      (\codetime,\baselineenergy)
    (0,0)};
    \addlegendentry{Runtime Optimisation}

    %Power Optimisation
    \addplot[area legend, fill=red, fill opacity=0.2, draw=none] coordinates {
                           (\codetime,\codeenergy)
                           (\pgfkeysvalueof{/pgfplots/xmax},\pgfkeysvalueof{/pgfplots/xmax}*\codepower)
                           (\pgfkeysvalueof{/pgfplots/xmax}, \pgfkeysvalueof{/pgfplots/xmax}*\baselinepower)
                           (0,0)
                         } ;
    \addlegendentry{Power Optimisation}
  
    %Energy Optimisation
    \addplot[area legend, style={pattern=north west lines, pattern color=gray,
                                 draw=none}] coordinates { 
      (\rooflinetime, \codeenergy)
      (\baselinetime, \codeenergy)
      (0, 0) };
    \addlegendentry{Energy Optimisation}



    % Constant Time, Energy Dashes
    %vertical
    \draw[densely dotted] ({axis cs:\codetime,\baselineenergy}) -- ({axis cs:\codetime,\rooflineenergy});
    %horizontal
    \draw[densely dotted] ({axis cs:\rooflinetime,\codeenergy}) -- ({axis cs:\baselinetime,\codeenergy});
   

    \node[circle,fill,inner sep=1pt] at (axis cs:\codetime,\codeenergy) {};
    \node[below right] at (axis cs:\codetime,\codeenergy) {$\theta$};


 \end{axis}
\end{tikzpicture}
