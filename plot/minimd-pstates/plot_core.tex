\@ifundefined{pstateminimdtable}{%
  \pgfplotstableread[col sep=comma]{plot/minimd-pstates/data/pstate_power_4_cores.csv}\pstateminimdtable
}{}

\begin{tikzpicture}
  \begin{axis}[ 
    axis on top,
    axis x line=bottom,
    axis y line=left,
  	xlabel={Runtime \emph{(s)}},
    ylabel={Energy \emph{(J)}},    
    xmin=0, xmax=200,
    ymin=0, ymax=1500,
    width=\linewidth,
    legend style={legend pos=north west}
    ]

    %% Model Parameters %%
    \pgfplotstablegetelem{0}{Runtime}\of{\pstateminimdtable}
    \pgfmathsetmacro{\codetime}{\pgfplotsretval} 
    \pgfplotstablegetelem{0}{Energy}\of{\pstateminimdtable}
    \pgfmathsetmacro{\codeenergy}{\pgfplotsretval} 
    \pgfmathsetmacro{\baselinepower}{26.614061}

    %% Intermezzo Values %%
    \pgfmathsetmacro{\codepower}{\codeenergy / \codetime}

    % arguments: code power, code time, x, n 
    \pgfmathdeclarefunction{metricbound}{4}{%
      \pgfmathparse{((#1 * #2^(#4 + 1)) / #3^#4)}%
    }
    \pgfmathdeclarefunction{definitionbound}{4}{%
      \pgfmathparse{((#1 / #2^(#4 + 1)) * #3^(#4 + 2))}%
    }

    % ALPHA BASELINE BOUND 
    \addplot[color=green, domain=\pgfkeysvalueof{/pgfplots/xmin}:\pgfkeysvalueof{/pgfplots/xmax}] {\baselinepower * x};
    \addlegendentry{$P_{\alpha}$ Energy Bound} 


    \addplot[domain=80:120, name path=basebound]{700};
    \addlegendentry{Baseline}

    \addplot[name path=edpdef, domain=80:\codetime, forget plot] {definitionbound(\codepower, \codetime, x, 1)};
    \addplot[name path=edpopt, domain=\codetime:120, forget plot] {metricbound(\codepower, \codetime, x, 1)}; 
    \path[name path=edpspace,
      intersection segments={
        of=edpdef and edpopt,
        sequence=A0 -- B1,
      }
    ]; 
    \addplot[blue!13] fill between[of=edpspace and basebound];
    \addlegendentry{$EDP$ Optimization Space}

%    \addplot[blue!6] fill between[of=edp and basebound];
    %Progression of PStates for MiniMD
    \addplot table[x=Runtime,y=Energy, trim cells=true] {\pstateminimdtable};
    \addlegendentry{MiniMD PState Progression}

 \end{axis}
\end{tikzpicture}
